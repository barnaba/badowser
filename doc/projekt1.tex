%\documentclass[a4paper]{scrartcl}
\documentclass[a4paper]{article}
\usepackage{fullpage}
\usepackage{mdwlist}
\usepackage{polski}
\usepackage[utf8x]{inputenc}
\usepackage{color}
\usepackage{graphicx}
\usepackage[unicode=true]{hyperref}
\usepackage{listings}

\lstset{ %
basicstyle=\footnotesize,       % the size of the fonts that are used for the code
numbers=left,                   % where to put the line\dywiz numbers
numberstyle=\footnotesize,      % the size of the fonts that are used for the line\dywiz numbers
stepnumber=2,                   % the step between two line\dywiz numbers. If it's 1, each line 
% will be numbered
numbersep=5pt,                  % how far the line\dywiz numbers are from the code
frame=single,                   % adds a~frame around the code
}

\newcommand{\doublesignature}[3][]{%
  \parbox{\textwidth}{
    \centering #3 \today\\
    \vspace{2cm}

    \parbox{7cm}{
      \centering
      \rule{6cm}{1pt}\\
       #1 
    }
    \hfill
    \parbox{7cm}{
      \centering
      \rule{6cm}{1pt}\\
      #2
    }
  }
}

\begin{document}
\sloppy

\title{BaDoWSer}
%\subtitle{silniczek graficzny w~coffeescripcie -- wirtualna kamera}
\author{Bartosz Pieńkowski, Barnaba Turek}
\maketitle
\abstract{
Celem projektu jest zapoznanie się z~rozproszonym otwartym system bazodanowym CouchDB.
Projekt podzielony jest na 6 etapów, każdy z~nich odpowiada konkretnemu zagadnieniu i
związane z~nim są konkretne artefakty.
}
\section{Motyw projektu}
Projekt będzie kopiował podstawową funkcjonalność serwisów mikroblogowych (takich jak twitter).
Danymi przechowywanymi w~bazie będą więc wiadomości, posiadane przez użytkowników i~ewentualnie połączone tagami.

\section{Etapy projektu}
\subsection{Rozpoznanie środowiska}
Celem etapu jest zapoznanie się z~podstawowymi funkcjami bazy danych, takimi jak dodawanie, usuwanie, wyszukiwanie i~przeglądanie danych.

\textbf{Artefakty:} Działający system zawierający jedną bazę danych, która bez żadnych dodatkowych programów jest w~stanie wyświetlić jakieś dane, zapisać jakieś dane i~w~jakiś sposób przetworzyć dane (np. znaleźć najnowsze wiadomości).

\subsection{Replikacja}
Celem etapu jest zapoznanie się z~tematem replikacji w~bazie, przećwiczenie różnych scenariuszy (Awaria części baz i~ponowne ich włączenie do systemu).

\textbf{Artefakty:} Działający system rozproszony na wielu maszynach (zarówno wirtualnych jak i~fizycznych - ponieważ projekt jest dwuosobowy nie mamy możliwosci dużego rozproszenia w~sensie geograficznym), odporny na awarię części baz.

\subsection{Replikacja II}
Celem etapu jest zapoznanie się z~zagadnieniem rozwiązywania konfliktów (które mogą wystąpić kiedy połączenie między bazami zostanie utracone).

\textbf{Artefakty:} Działający system rozproszony na wielu maszynach, który nie przestaje działać, jeśli maszyny stracą między sobą połączenie. Po ponownym uzyskaniu połączenia system próbuje połączyć dane i~istnieją scenariusze, w~których mu się udaje. System jest też odporny na jednoczesne edytowanie tych samych danych. 

\subsection{Przetwarzanie danych}
Celem etapu jest zagłębienie się w~tematy związane z~przetwarzaniem danych w~systemie opartym o~couchDB. Tematy takie jak indeksowanie i~budowanie widoków.

\textbf{Artefakty:} Element systemu, który przetwarza dane, a~nie tylko je wyświetla. Przykładem takiego elementu może być wyszukiwarka lub komponent generujący statystyki.

\subsection{Rozproszone migracje}
Celem etapu jest zapoznanie się z~tematem zmiany schematu danych i~propagacji tej zmiany w~działącym środowisku.

\textbf{Artefakty:} Zmiana elementu systemu i~struktury bazy, taka jak dodanie relacji (np. tagi, albo przynależność postów do użytkowników)

\subsection{Lokalny klient}
Celem etapu jest zapoznanie się z~opcją kopiowania bazy aplikacji do lokalnie działającej bazy danych - tak aby każdy użytkownik aplikacji mógł z~niej korzystać, nawet kiedy straci połączenie z~bazą.

\textbf{Artefakty:} System pozwalający na bardzo dynamiczną i~często zmianę ilości podłączonych baz (odporny na to, że użytkownicy wyłączają komputery. 

\end{document}
